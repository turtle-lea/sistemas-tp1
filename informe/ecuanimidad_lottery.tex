\section{Ecuanimidad del SchedLottery}

En el ejercicio 5 implementamos un scheduler en base al paper ''Lottery scheduling''. Intentamos generar un mecanismo de control eficiente, flexible y justo en donde cada tarea fuese procesada
aproximadamente un tiempo equivalente a las otras, sin importar su tipo. Este último aspecto es el que nos permite hablar de ecuanimidad del scheduler.

Para ejemplificar lo mencionado anteriormente, supongamos que en una computadora de un solo core se han lanzado dos procesos que corren en simultáneo. El primero de ellos, al cual llamaremos $A$, utiliza
activamente el CPU sin necesidad de hacer llamados al sistema o esperar que se liberen recursos de la máquina (podría ser, por ejemplo, el caso de una aplicación científica que ejecuta gran cantidad de 
cuentas). Por otro lado tenemos a la tarea $B$, que ejecuta una instrucción bloqueante cada cierto tiempo, desbloquéandose luego de una cierta cantidad de ticks del reloj. Supongamos también que ambos conviven
con otros procesos que también están siendo ejecutados por el cpu. En el caso de un scheduler Round Robin, a lo largo de una ronda se le asigna un quantum a cada proceso. Sin embargo, notemos que mientras
la tarea $A$ utiliza la totalidad del quantum que le han asignado, $B$ usa únicamente una fracción: $\frac{f}{q}$. De esta manera, luego de sucesivas rondas la tarea $A$ ha sido ejecutada un tiempo
considerablemente mayor al de $B$.

Para solucionar este problema, la idea es que el scheduler seleccione la próxima a ser ejecutada mediante el sorteo de una lotería, en donde aquellas tareas que utilicen solamente una fracción del quantum
tengan más probabilidades de ser las ganadoras. Para lograr este objetivo, utilizamos un sistema de ''compensation tickets'', que funciona de la siguiente manera:

- Las tareas que llegan por primera vez entran con una cantidad equivalente a $\frac{tickets}{1 \ quantum}$. Por ejemplo, si el 
- Las tareas que utilizaron la totalidad del quantum la último





 
