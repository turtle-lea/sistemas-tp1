El siguiente trabajo tiene como objetivo analizar distintos algoritmos de $scheduling$, evaluándolos sobre distintos lotes de tareas. 

Para construir lotes variados implementamos diferentes tareas que simulan los distintos tipos de ellas, sean de procesamiento intenso o interactivas, etc. La idea es construir lotes con los cuales analizar el comportamiento de los distintos $schedulers$ simulando posibles diferentes arquitecturas, por ejemplo lotes con muchas tareas interactivas podrían ser propios de celulares, de una PC combinando tareas de procesamiento intenso. Incluso podríamos querer simular máquinas de mucho procesamiento de datos.

Se implementaron cuatro algoritmos de $scheduling$ y se evaluó su desempeño sobre los diferentes lotes de tareas mencionados anteriormente en base a dos métricas: Turnaround time y Waiting time.

Para un mejor análisis, la cátedra nos facilitó un graficador para el status de las tareas y los nucleos de procesamiento, que toma de entrada el output del simulador $simusched$ (también provisto por la cátedra) el cual utiliza los algoritmos de $scheduling$ programados por nosotros.

El trabajo fue dividido en tres partes:
