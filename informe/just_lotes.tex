Luego de verificar el correcto funcionamiento de los algoritmos de $scheduling$ que hicimos con peque�os lotes de tareas, y con el fin de comparar y analizar m�s profundamente los distintos algoritmos, decidimos generar lotes principales m�s grandes y organizados que intenten simular los distintos lotes t�picos de distintas arquitecturas.

As� surgi� el $generador\_lotes.sh$ que produce cuatro lotes a conocer:

\begin{itemize}
	\item $loteCelular.tsk$ posee un total de 40 tareas interactivas (taskConsola), simula un ambiente interactivo que podr�a entenderse como acceso a internet o comunicaci�n con otro dispositivo, etc.
	\item $lotePC.tsk$ posee un total de 60 tareas de distintos tipos (bloqueantes, uso intenso cpu, etc.), dado el amplio uso que puede otorgarse a una PC. Consideramos que aunque 60 parece ser un n�mero bajo para una PC promedio (visto que se suelen encontrar entre 150 y 200 procesos corriendo en una PC con el comando \textbf{ps -ef | wc -l} en linux), es un n�mero que permite un buen an�lisis y a la vez simplifica el tiempo de c�mputo.
	\item $loteCalc.tsk$ est� compuesto por 200 tareas de procesamiento intenso.
	\item $loteRTC.tsk$ simula una arquitectura de $real-time computing$ o c�mputo en tiempo real, con un total de 100 taskBatch dado que es tan importante el procesamiento intenso como la interactividad.
\end{itemize}

