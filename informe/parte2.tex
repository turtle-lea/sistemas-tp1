Una vez que entendemos el simulador y usamos el graficador con un scheduler básico como lo es SchedFCFS nos proponemos a realizar nuevos schedulers: \textbf{SchedRR} y \textbf{SchedLottery}.

\subsection{SchedRR}

El siguiente paso fue completar la implementación de SchedRR, el cual se basa en el algoritmo de scheduling $round-robin$. 

Básicamente las tareas forman una ronda para correr en algún núcleo (de haber más de uno). Cada CPU tiene asociado un quantum máximo, es decir, una cantidad máxima de ciclos en el cual puede correr una determinada tarea. El quantum de cada CPU es parámetro de nuestro scheduler.

Así como el SchedFCFS, éste scheduler esta implementado sobre una cola de tareas global (para permitir migración), pero a diferencia del anterior, nuestro scheduler irá encolando a las tareas nuevamente en la cola una vez que sean desalojadas del CPU, ya sea por realizar una llamada bloqueante o debido a que su quantum se completó. Si la tarea se bloqueo, no sera encolada en la cola de tareas $ready$ hasta que se desbloquee, por lo cual la función UNBLOCK comienza a tener sentido y es ahi donde la tarea regresa a la cola.

Para determinar si una tarea cumplió con el quantum máximo del CPU en la cual corre tenemos un vector de quantums parciales, un contador para cada CPU. Por cada tick se incrementa en uno y se resetea en caso de que la tarea sea desalojada. Notar que si la tarea se bloquea es desalojada y por lo tanto pierde su quantum.

Como detalle de implementación determinamos que si alguna tarea se bloquea y no hay ninguna tarea lista para correr, es decir, la cola de tareas $ready$ esta vacia, ésta seguirá corriendo en el CPU para evitar pagar costos de cambio de contexto.

Veamos el gráfico a continuación:

\begin{figure}[H]
\centering\includegraphics[width=15 cm]{graficos/ej4RR1.png}
\centering\includegraphics[width=15 cm]{graficos/ej4RR2.png}
\caption{Round-robin, un sólo núcleo de procesamiento en el primer caso, 4 en el segundo. Distintos lotes de tareas.}
\end{figure}

Se confeccionaron dos lotes diferentes para la figura 2. Para el primer caso se corrieron dos TaskCPU, una TaskConsola y dos TaskAlterno (alternan entre uso de CPU y llamadas bloqueantes). Como el primer gráfico muestra el comportamiento con un único núcleo es facil ver como SchedRR funciona. En orden de llegada cada tarea corre su quantum, o lo pierde si se bloquea para que otra tome su lugar.

El segundo caso refleja el mismo funcionamiento pero quad-core, con iguales quantums máximos para cada CPU. En este caso el lote estaba compuesto por seis TaskCPU y sólo una TaskConsola. Como en el ejemplo no hay costos de migración, algunas tareas terminan el quantum de un núcleo y automáticamente corren en otro, pero es de esperarse. La TaskConsola se bloquea y libera el CPU para otras tareas.

\subsection{SchedLottery y compensaciones probabilísticas}

\subsubsection{Ecuanimidad SchedLottery con compensation tickets}

SchedLottery fue testeado exhaustivamente y al utilizar el sistema de compensation tickets se comporta de forma esperada. Entre toda la experimentación
decidimos quedarnos con un par de muestras que exhiben aspectos relevantes de su comportamiento.

La siguiente figura muestra un experimento corrido sobre un scheduler con lote de 5 tareas, en donde 1 es bloqueante y las otras consumen constantemente CPU. 
El quantum del mismo es de 4 ticks, por lo que luego de las primeras corridas de cada una tendríamos: 

tarea bloqueante: 4 tickets. Resto de las tareas: 1 ticket. Cantidad de tickets totales en circulación: 8. 

Por lo tanto, lo esperable es que cada 8 sorteos en donde participen las 5 tareas (tengamos en cuenta que cuando una tarea se bloquea una cantidad de clocks $\neq \ 0$ no 
participa en el próximo sorteo), 4 sean ganadas por la tarea bloqueante, y el resto de las tareas gane una única vez.

\begin{figure}[!h]
	\begin{center}
		  \includegraphics[scale=0.5]{Graficos/intervalo_8.png}
		  \caption{Comportamiento de SchedLottery para 5 tareas: 1 bloqueantes y 4 tareas CPU }
		  \label{fig:contra1}
	\end{center}
\end{figure}
\FloatBarrier

Efectivamente, utilizamos esta figura para mostrar que nuestras hipótesis se confirmaron (el mismo experimento fue corrido repetidas veces arrojando resultados similares).
Si consideramos los sorteos en donde participan las 5 tareas, en el intervalo de tiempo encerrado por el rectángulo azul, observamos como la tarea bloqueante sale
ganadora en 4 de las 8 ocasiones. La tarea 2 gana en 2 ocasiones. La tarea 4 y la 1 ganan en una ocasión (lo esperable), mientras que la tarea 3 no sale sorteada ninguna vez.

\subsection{Experimentos}

Para verificar el $fairness$ de nuestro scheduler decidimos cargarle una determinada cantidad de tareas y comparar cuántos ticks insumía cada una en un rango determinado. Si efectivamente existe ecuanimidad, cada una de las tareas debería insumir una cantidad similar de ticks.

Para testear este comportamiento tomamos el siguiente lote de tareas: una bloqueante y tres que consumen cpu. Las tres insumen en total una cantidad fija de cilos. La bloqueante bloquea 16 veces y las demás tienen un quantum de 32. Como se trata de un scheduler $pseudoaleatorio$ fue necesario realizar el expermimento varias veces con el objetivo de evitar resultados indeseados.

Decidimos tomar como rango inicial el instante en el que todas las tareas ya fueron ejecutadas ya que a partir de allí comenzarán a influir los $compensations \ tickets$. El rango final se determinó de forma arbitraria cersiorándose de que se encuentre antes de la finalización de alguna tarea.

Realizamos el experimento para un scheduler de 1 core y un quantum de 4 y los resultados fueron los siguientes:

\begin{figure}[!h]
	\begin{center}
		  \includegraphics[scale=0.3]{Graficos/comp1.png}
		  \caption{Ticks insumidos por cada tarea para un schedLottery de 4 tareas: 1 bloqueante y 3 de CPU (1 core)}
		  \label{fig:contra1}
	\end{center}
\end{figure}
\FloatBarrier

Para 2 cores ocurrió lo siguiente:

\begin{figure}[!h]
	\begin{center}
		  \includegraphics[scale=0.3]{Graficos/comp2.png}
		  \caption{Ticks insumidos por cada tarea para un schedLottery de 4 tareas: 1 bloqueante y 3 de CPU (2 core) }
		  \label{fig:contra1}
	\end{center}
\end{figure}
\FloatBarrier

Como se puede observar, los ticks insumidos por cada tarea es bastante equitativa. Esto quiere decir que los $compensation \ tickets$ están actuando a favor de la tarea bloqueante que es la que utiliza solo una parte de su quantum cada vez que se ejecuta.

Los $compensation \ tickets$ son los que lo otorgan equanimidad a este scheduler. Sin ellos la tarea bloqueante no hubiera insumido los ticks que insumió en las pruebas anteriores. Para comprobar esto, corrimos el mismo scheduler con el mismo lote de tareas pero sin $compensation \ tickets$. Estos fueron los resultados:

Para 1 core:

\begin{figure}[!h]
	\begin{center}
		  \includegraphics[scale=0.3]{Graficos/sincomp1.png}
		  \caption{Ticks insumidos por cada tarea para un schedLottery de 4 tareas: 1 bloqueante y 3 de CPU (1 core). Sin $compensation \ tickets$}
		  \label{fig:contra1}
	\end{center}
\end{figure}
\FloatBarrier

Para 2 cores:

\begin{figure}[!h]
	\begin{center}
		  \includegraphics[scale=0.3]{Graficos/sinComp2.png}
		  \caption{Ticks insumidos por cada tarea para un schedLottery de 4 tareas: 1 bloqueante y 3 de CPU (1 core). Sin $compensation \ tickets$}
		  \label{fig:contra1}
	\end{center}
\end{figure}
\FloatBarrier

Como puede oservarse, los ticks insumidos por la tarea bloqueante es mucho menor de lo que era antes. La ecuanimidad ya no está presente en este scheduler.

