Una vez que entendemos el simulador y usamos el graficador con un scheduler básico como lo es SchedFCFS nos proponemos a realizar nuevos schedulers: \textbf{SchedRR} y \textbf{SchedLottery}.

\subsection{SchedRR}

El siguiente paso fue completar la implementación de SchedRR, el cual se basa en el algoritmo de scheduling $round-robin$. 

Básicamente las tareas forman una ronda para correr en algún núcleo (de haber más de uno). Cada CPU tiene asociado un quantum máximo, es decir, una cantidad máxima de ciclos en el cual puede correr una determinada tarea. El quantum de cada CPU es parámetro de nuestro scheduler.

Así como el SchedFCFS, éste scheduler esta implementado sobre una cola de tareas global (para permitir migración), pero a diferencia del anterior, nuestro scheduler irá encolando a las tareas nuevamente en la cola una vez que sean desalojadas del CPU, ya sea por realizar una llamada bloqueante o debido a que su quantum se completó. Si la tarea se bloqueo, no sera encolada en la cola de tareas $ready$ hasta que se desbloquee, por lo cual la función UNBLOCK comienza a tener sentido y es ahi donde la tarea regresa a la cola.

Para determinar si una tarea cumplió con el quantum máximo del CPU en la cual corre tenemos un vector de quantums parciales, un contador para cada CPU. Por cada tick se incrementa en uno y se resetea en caso de que la tarea sea desalojada. Notar que si la tarea se bloquea es desalojada y por lo tanto pierde su quantum.

Como detalle de implementación determinamos que si existe una única tarea y ésta se bloquea seguirá corriendo en el CPU para evitar pagar costos de cambio de contexto.

Veamos el gráfico a continuación:

\begin{figure}[H]
\centering\includegraphics[width=18 cm]{graficos/ej4RR1.png}
\centering\includegraphics[width=18 cm]{graficos/ej4RR2.png}
\caption{First-Come First-Served, un sólo núcleo de procesamiento en el primer caso, dos en el segundo y por último tres núcleos.}
\end{figure}



\subsection{SchedLottery y compensaciones probabilísticas}

\subsubsection{Ecuanimidad SchedLottery con compensation tickets}

SchedLottery fue testeado exhaustivamente y al utilizar el sistema de compensation tickets se comporta de forma esperada. Entre toda la experimentación
decidimos quedarnos con un par de muestras que exhiben aspectos relevantes de su comportamiento.

La siguiente figura muestra un experimento corrido sobre un scheduler con lote de 5 tareas, en donde 1 es bloqueante y las otras consumen constantemente CPU. 
El quantum del mismo es de 4 ticks, por lo que luego de las primeras corridas de cada una tendríamos: 

tarea bloqueante: 4 tickets. Resto de las tareas: 1 ticket. Cantidad de tickets totales en circulación: 8. 

Por lo tanto, lo esperable es que cada 8 sorteos en donde participen las 5 tareas (tengamos en cuenta que cuando una tarea se bloquea una cantidad de clocks $\neq \ 0$ no 
participa en el próximo sorteo), 4 sean ganadas por la tarea bloqueante, y el resto de las tareas gane una única vez.

\begin{figure}[!h]
	\begin{center}
		  \includegraphics[scale=0.5]{Graficos/intervalo_8.png}
		  \caption{Comportamiento de SchedLottery para 5 tareas: 1 bloqueantes y 4 tareas CPU }
		  \label{fig:contra1}
	\end{center}
\end{figure}
\FloatBarrier

Efectivamente, utilizamos esta figura para mostrar que nuestras hipótesis se confirmaron (el mismo experimento fue corrido repetidas veces arrojando resultados similares).
Si consideramos los sorteos en donde participan las 5 tareas, en el intervalo de tiempo encerrado por el rectángulo azul, observamos como la tarea bloqueante sale
ganadora en 4 de las 8 ocasiones. La tarea 2 gana en 2 ocasiones. La tarea 4 y la 1 ganan en una ocasión (lo esperable), mientras que la tarea 3 no sale sorteada ninguna vez.

