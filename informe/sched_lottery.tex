\section{SchedLottery}

En el ejercicio 5 implementamos un scheduler en base al paper ''Lottery scheduling''. Intentamos generar un mecanismo de control eficiente, flexible y justo en donde cada tarea fuese procesada
aproximadamente un tiempo equivalente a las otras, sin importar su tipo. Este último aspecto es el que nos permite hablar de ecuanimidad del scheduler.

Para ejemplificar lo mencionado anteriormente, supongamos que en una computadora de un solo core se han lanzado dos procesos que corren en simultáneo. El primero de ellos, al cual llamaremos $A$, utiliza
activamente el CPU sin necesidad de hacer llamados al sistema o esperar que se liberen recursos de la máquina (podría ser, por ejemplo, el caso de una aplicación científica que ejecuta gran cantidad de 
cuentas). Por otro lado tenemos a la tarea $B$, que ejecuta una instrucción bloqueante cada cierto tiempo, desbloquéandose luego de una cierta cantidad de ticks del reloj. Supongamos también que ambos conviven
con otros procesos que también están siendo ejecutados por el cpu. En el caso de un scheduler Round Robin, a lo largo de una ronda se le asigna un quantum a cada proceso. Sin embargo, notemos que mientras
la tarea $A$ utiliza la totalidad del quantum que le han asignado, $B$ usa únicamente una fracción: $\frac{f}{q}$. De esta manera, luego de sucesivas rondas la tarea $A$ ha sido ejecutada un tiempo
considerablemente mayor al de $B$.

Para solucionar este problema, la idea es que el scheduler seleccione la próxima a ser ejecutada mediante el sorteo de una lotería, en donde aquellas tareas que utilicen solamente una fracción del quantum
tengan más probabilidades de ser las ganadoras. Para lograr este objetivo, utilizamos un sistema de ''compensation tickets'', que funciona de la siguiente manera:

- Las tareas que llegan por primera vez entran con una cantidad equivalente a $\frac{\#tickss}{quantum}$. Por ejemplo, si cada quantum consta de 8 ticks, entonces cada
proceso recibirá 8 tickets iniciales, con lo cual evitamos que las tareas recién ingresadas sufran de inanición (se les dé la oportunidad de correr para conocer su tipo, 
y comepensarlos (o no) con la cantidad de tickets correspondiente).

- Cuando una tarea es desalojada debido a una llamada bloqueante, lo más probable es que solamente haya utilizado una fracción del quantum. Sea esta última $\frac{f}{q}$,
en donde q es igual a $\frac{\#tickss}{quantum}$ y f la cantidad de ticks consumidos por la tarea durante el último llamado.
Anteriormente habíamos dicho que el scheduler debía encargarse de que todas las tareas corrieran una cantidad de tiempo lo más similar posible a las demás, y que para ello utilizaba
el sistema de compensation tickets. Nuestro mecanismo hace que la tarea ingrese a la cola de Ready's con $\frac{q}{f}$ tickets, con lo cual tiene una probabilidad de ganar
inversamente proporcional a la fracción de quantum que está usando.

- Cuando se produce un tick y el scheduler detecta que la tarea consumió la totalidad de su quantum, la misma es desalojada y encolada en el vector de Ready's con un solo ticket.

~
Poniendo en un ejemplo lo anteriormente mencionado, supongamos una máquina con un scheduler con
un quantum de 8 ticks manejando temporalmente 
1 tarea bloqueante $A$ (cuyos bloqueos duran 1 ciclo), y 2 tareas $B$ y $C$ que consumen CPU. En un principio todas entran al sorteo con 8
tickets. Sin embargo, luego de que las tres han corrido por lo menos una vez, siempre entrarán al sorteo con una misma cantidad: $A=8, \ B=1, \ C=1$, con un total de
10 tickets en circulación. Con estos valores, se espera que cada 10 sorteos (número que coincide con la cantidad de tickets en circulación): 8 sean ganados por A,
1 sea ganado por B y 1 por C, con lo cual todas estarían corriendo la misma cantidad de ticks en el cpu. 




 
